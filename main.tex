\documentclass{article}
\usepackage[utf8]{inputenc}

\title{CoppeliaSim}
\author{Ruiming Guo}
\date{\today}

\usepackage{listings}

\begin{document}

\maketitle

\section{Introduction}

The robot simulator, CoppeliaSim, is an integrated robot developing environment, with physical simulation, scripting robot control and a friendly user interface.

The main scene and each of the robots can be controlled by scripts.

In this project, we use CoppeliaSim to simulate the Fully Automated Garbage Sorting System (FAGSS).

The CoppeliaSim lib is released under GNU GPL, while other accompanying codes that have other open-source or proprietary licenses. 

CoppeliaSim can be used as a stand-alone application or can easily be embedded into a main client application. But as for now, we'll just use it as it is.

\section{scripts}

The main thing we care about, which is also why CoppeliaSim so famous, is its scripting capblility. 

\emph{Simulation Scripts} plays the central role in a simulation. There are two types of Simulation Scripts: the \emph{main script}, and the \emph{child script}.

\subsection{The main script}

By default, each scene in CoppeliaSim will have one main script, which contains the basic code that allows a simulation to run.

The main script is a collection of \emph{callback functions}. Among all these callbacks, \txttt{sysCall_init} is mandate, while others are optional.

But most of the time we needn't, and shouldn't modify this script.

Most of child's script's system callback functions are called from the main script, which operates in a \textbf{cascading} fashion upon the scene hierarchy and the child scripts attached to individual scene object. The leaf's callbacks are executed first, and the root's last.

\end{document}
